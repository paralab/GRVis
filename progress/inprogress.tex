In order to complete this project, we still need to do the following things.

\begin{itemize}
	\item \textbf{Incorporate VTK filters into parallel Python script.} Currently the script loads the data from .pvtu files and prepares them for processing. Now we need to incorporate the VTK filters to prepare the final visualizations.
	\item \textbf{Recreate images/videos generated by old ParaView script.} Once the VTK filters are incorporated, we can re-generate all of the images and videos from our old ParaView script and compare the two methods.
	\item \textbf{Combine VTK rendering process with existing C++ code.} Since the inevitable goal of this project is to render the relevant images/videos as the simulation runs, we need to create either a C++ program to do the rendering or an interface between python and the original C++ simulation. The previous two in-progress items are quick to accomplish but this item will take the majority of our time.
\end{itemize}