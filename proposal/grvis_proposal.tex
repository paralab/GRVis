\documentclass[a4paper,10pt]{article}
\usepackage[utf8]{inputenc}
\usepackage[margin=1.25in]{geometry}
\usepackage{setspace}
\newcommand{\grvis}{\texttt{GRVis}}
\doublespace
\title{\grvis : Visualization Framework for Computational General Relativity}
\author{Milinda Fernando (u1011531), Max Carlson (u0450449)}

\begin{document}

\maketitle

\section{Introduction}
%Give an overview of the project.
%Why is this project important and/or interesting
\input{intro}

%What data will you be using for your project
\section{Data}
The PDEs are discretized in both space and time, and we are marching in time computing how the solution evolves through time. Hence the data is time dependent. Dendro uses \texttt{.pvtu} file format to write the solution at a given time step. The \texttt{.pvtu} file format specifies the partitioned unstructured grids where it contains information such as number of partitions and file names of those partitions in order to facilitate parallel rendering of the data. Each partition is in \texttt{.vtu} file format written in binary format with compression enabled. An example \texttt{.pvtu} file is given below which list of variables written and file name for each partition (i.e. \texttt{.vtu} files). 

\begin{lstlisting}[language=Xml]
<?xml version="1.0"?>
<VTKFile type="PUnstructuredGrid" version="0.1" byte_order="LittleEndian">
<PUnstructuredGrid GhostLevel="0">
<PFieldData>
<PDataArray type="Float32" Name="Time" format="ascii"/>
<PDataArray type="Float32" Name="Cycle" format="ascii"/>
</PFieldData>
<PPoints>
<PDataArray type="UInt32" Name="Position" NumberOfComponents="3" format="binary"/>
</PPoints>
<PCellData>
<PDataArray type="UInt32" Name="mpi_rank" format="binary"/>
<PDataArray type="UInt32" Name="cell_level" format="binary"/>
</PCellData>
<PPointData>
<PDataArray type="Float64" Name="U_ALPHA" format="binary"/>
<PDataArray type="Float64" Name="U_CHI" format="binary"/>
<PDataArray type="Float64" Name="U_K" format="binary"/>
<PDataArray type="Float64" Name="U_GT0" format="binary"/>
<PDataArray type="Float64" Name="U_GT1" format="binary"/>
<PDataArray type="Float64" Name="U_GT2" format="binary"/>
<PDataArray type="Float64" Name="U_BETA0" format="binary"/>
<PDataArray type="Float64" Name="U_BETA1" format="binary"/>
<PDataArray type="Float64" Name="U_BETA2" format="binary"/>
<PDataArray type="Float64" Name="U_B0" format="binary"/>
<PDataArray type="Float64" Name="U_B1" format="binary"/>
<PDataArray type="Float64" Name="U_B2" format="binary"/>
<PDataArray type="Float64" Name="U_SYMGT0" format="binary"/>
<PDataArray type="Float64" Name="U_SYMGT1" format="binary"/>
<PDataArray type="Float64" Name="U_SYMGT2" format="binary"/>
<PDataArray type="Float64" Name="U_SYMGT3" format="binary"/>
<PDataArray type="Float64" Name="U_SYMGT4" format="binary"/>
<PDataArray type="Float64" Name="U_SYMGT5" format="binary"/>
<PDataArray type="Float64" Name="U_SYMAT0" format="binary"/>
<PDataArray type="Float64" Name="U_SYMAT1" format="binary"/>
<PDataArray type="Float64" Name="U_SYMAT2" format="binary"/>
<PDataArray type="Float64" Name="U_SYMAT3" format="binary"/>
<PDataArray type="Float64" Name="U_SYMAT4" format="binary"/>
<PDataArray type="Float64" Name="U_SYMAT5" format="binary"/>
</PPointData>
<Piece Source="bssn_gr_0_0_16.vtu"/>
<Piece Source="bssn_gr_0_1_16.vtu"/>
<Piece Source="bssn_gr_0_2_16.vtu"/>
<Piece Source="bssn_gr_0_3_16.vtu"/>
<Piece Source="bssn_gr_0_4_16.vtu"/>
<Piece Source="bssn_gr_0_5_16.vtu"/>
<Piece Source="bssn_gr_0_6_16.vtu"/>
<Piece Source="bssn_gr_0_7_16.vtu"/>
<Piece Source="bssn_gr_0_8_16.vtu"/>
<Piece Source="bssn_gr_0_9_16.vtu"/>
<Piece Source="bssn_gr_0_10_16.vtu"/>
<Piece Source="bssn_gr_0_11_16.vtu"/>
<Piece Source="bssn_gr_0_12_16.vtu"/>
<Piece Source="bssn_gr_0_13_16.vtu"/>
<Piece Source="bssn_gr_0_14_16.vtu"/>
<Piece Source="bssn_gr_0_15_16.vtu"/>
</PUnstructuredGrid>
</VTKFile> 
\end{lstlisting}


% What are the objectives of the project? What are the questions you want to answer?
% What would you like to learn by completing this projec
\section{Objectives}
The primary objective of this project is to develop a parallel rendering pipeline to quickly and easily generate animations from the data generated by the black hole simulation. Since the data has up to 24 output fields, analyzing the results is impractical without some form of visualization. This simulation is also of a very large size and can be run on thousands of nodes which can be a nightmare to debug. With a solid visualization pipeline, errors and unexpected behavior can be seen much more easily.

Since the work in this class is largely done using ParaView, and both of us have some experience rendering using parallel ParaView, this is also a good chance to compare ParaView and Visualization ToolKit (VTK) \cite{vtk}. For instance, knowing that ParaView is an extension of VTK, how much computational overhead does ParaView add? Does VTK make rendering in parallel easier or harder? These are two of the questions we intend to explore during the implementation of this project.

Finally, once we have a visualization pipeline in place, we can explore an advanced topic related to visualization such as In Situ Rendering, Fourth Order Tensor Visualization or other topics that may present themselves naturally during the course of implementation. This phase of the project will give us the opportunity to explore the data and perhaps find structures that are rarely visualized or potentially unknown.

%What is your project schedule? What have you done thus far and what will you have to do to
%complete this project? Be as specific as possible.
\section{Schedule}
For the given time frame, we will break up the project into two phases. The first phase will consist of getting the python parallel rendering pipeline operational and will take place starting from March 15th until April 3rd when the first progress report is due. Then the second phase, starting April 3rd until the due date April 19th, will involve extending this pipeline to cover an "advanced" topic which will be decided on when writing the progress report.

For phase 1, the tentative schedule is

\begin{itemize}
	\item \textbf{March 15th through March 18th:} Get sequential VTK rendering examples running using single pvtu files from black hole simulation using python.
	\item \textbf{March 19th through March 23rd:} Extend the python script to render many pvtu files in parallel and stitch them together into a single image.
	\item \textbf{March 26th through March 30th:} Using the parallel python script, generate a few movies from the black hole data set. (single slice with warp by scalar, volume rendering, both combined)
	\item \textbf{April 2nd and April 3rd:} Write up project progress report and decide on focus for part 2.
\end{itemize}

Once phase 1 is complete, we will choose an advanced topic to explore and implement. This topic will be chosen during the progress report write-up and will include an updated schedule for the final April 3rd to April 19th time frame Some of the potential advanced topics include

\begin{itemize}
	\item \textbf{In Situ Rendering:} Extend the GR code to use VTK to render the frames from memory to bypass file IO. Drastically reduces amount of data that needs to be stored and the data is already naturally partitioned among the same number of nodes and processes that generated it.
	\item \textbf{Fourth Order Tensor Visualization:} Using a tensor decomposition such as tucker decomposition, convert the Ricci fourth order tensors into a form that can be visualized. If the tensors can be reduced to a collection of vector fields, a multi-field flow visualization would be very interesting to see.
\end{itemize}

%When the project is completed, how specifically can we evaluate how successful it is?
\section{Evaluation}
For phase 1 of the project, we will submit the python script that, in parallel, generates a set of animatinos from collections of \texttt{.pvtu} files for each time step. In addition, a selection of the rendered animations will be submitted with a written comparison of the differences between ParaView and VTK parallel rendering.

Then for phase 2, when an advanced topic has been decided upon, the final report will include a detailed writeup of what we learned during the implementation of the C++ pipeline and the advanced topic. This submission will include the C++ program that incorporates what we learned from the python script and any new rendered animations that arise from implementing the advanced topic. The final report will then also include a comparison of the python and C++ implementations.

\bibliographystyle{unsrt}%Used BibTeX style is unsrt
\bibliography{grvis}

\end{document}
