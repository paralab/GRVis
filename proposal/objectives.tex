The primary objective of this project is to develop a parallel rendering pipeline to quickly and easily generate animations from the data generated by the black hole simulation. Since the data has up to 24 output fields, analyzing the results is impractical without some form of visualization. This simulation is also of a very large size and can be run on thousands of nodes which can be a nightmare to debug. With a solid visualization pipeline, errors and unexpected behavior can be seen much more easily.

Since the work in this class is largely done using ParaView, and both of us have some experience rendering using parallel ParaView, this is also a good chance to compare ParaView and Visualization ToolKit (VTK) \cite{vtk}. For instance, knowing that ParaView is an extension of VTK, how much computational overhead does ParaView add? Does VTK make rendering in parallel easier or harder? These are two of the questions we intend to explore during the implementation of this project.

Finally, once we have a visualization pipeline in place, we can explore an advanced topic related to visualization such as In Situ Rendering, Rank-4 Tensor Visualization or other topics that may present themselves naturally during the course of implementation. This phase of the project will give us the opportunity to explore the data and perhaps find structures that are rarely visualized or potentially unknown.